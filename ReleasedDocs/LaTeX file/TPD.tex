\documentclass{article}
\usepackage[utf8]{inputenc}
\usepackage{graphicx}
\usepackage{float}
\usepackage{mathtools}

\title{Test plan document: bike4share}
\author{Maffioli Sara, Papale Lorenzo, \\ Stucchi Lorenzo \& Vaghi Federica}


\begin{document}
\maketitle
\tableofcontents

\newpage

\section{User Registration}
\subsection{Introduction}
The test is done to test the function of the registration form for a normal user of the “bike4share” webapp.
\subsection{Test Items}
\begin{itemize}
    \item Registration form on the page /register
\end{itemize}
\subsection{Features to be Tested}
\begin{itemize}
    \item The registration is done only if the username the password and the e-mail address are written
    \item The registration doesn’t works for an already existed username or an already existed e-mail address
\end{itemize}
\subsection{Features Not to Be Tested}
\begin{itemize}
    \item A minimum number of charter for username and password is not required. 
    \item the domain of the e-mail
\end{itemize}
\subsection{Approach}
\begin{itemize}
    \item The test is done manually. And the different combination described into the schedule are done. 
\end{itemize}
\subsection{Item Pass/Fail Criteria}
\begin{itemize}
    \item The test is passed if only the user with a new username, a password and an not already used e-mail is registered into the database of bike\textunderscore users and the user is registrated with user\textunderscore type equal to u (normal user). 
\end{itemize}
\subsection{Schedule}
\begin{itemize}
    \item The form is filled without putting a password. Only username “test 1” and a not already existed e-mail address are filled.
    \item The form is filled without putting a username. Only password “testpa” and a not already existed e-mail address are filled.
    \item The form is filled with username “test 1” and password “testpa” and an not already existed e-mail address.
    \item The form is filled with an existing username “ test1” .
    \item The form is filled with an already existed e-mail address.
\end{itemize}
\subsection{Final State}
\begin{itemize}
    \item The test is done manually and all the features tested works.
\end{itemize}

\section{Technician Registration}
\subsection{Introduction}
The test is done to test the function of the registration form for a technician user of the “bike4share” web application.
\subsection{Test Items}
\begin{itemize}
    \item Registration form on the page /tec\textunderscore reg
\end{itemize}
\subsection{Features to be Tested}
\begin{itemize}
    \item The registration is done only if the secret code, the username the e-mail and the password are written and the secret code, the username and the e-mail are not used  yet.
    \item The registration doesn’t works for an already used secret code, an existed username and or an existed e-mail address
\end{itemize}
\subsection{Features Not to Be Tested}
\begin{itemize}
    \item A minimum number of character for username and password is not required. 
\end{itemize}
\subsection{Approach}
\begin{itemize}
    \item The test is done manually. And the different combination described into the schedule are done. 
\end{itemize}
\subsection{Item Pass/Fail Criteria}
\begin{itemize}
    \item The test is passed if only the user with an never used secret code, a new username, a new e-mail address and a password is registered into the database of bike\textunderscore users, the user is registered with user\textunderscore type equal to t (technician user) and the secret code is deleted from the database secret\textunderscore key.
   \end{itemize}
\subsection{Schedule}
\begin{itemize}
    \item The form is filled without putting a secret code. Only username “tec1”, password “tecpa” and a not existing e-mail address.
    \item The form is filled without putting a password. Only secret code, username “tec1”and not existing e-mail address.
    \item The form is filled without putting a username. Only secret code password “tecpa”and a not existing e-mail address
    \item The form is filled without an e-mail address. Only secret code, username “tec1” and password “tecpa”.
    \item The form is filled with an existing username “ test1”.
    \item The form is filled with an invented secret code.
    \item The form is filled with an already used secret code.
     \item The form is filled with an existing e-mail address
\end{itemize}
\subsection{Final State}
\begin{itemize}
    \item The test is done manually and all the features tested works.
\end{itemize}

\section{Forgot Password}
\subsection{Introduction}
The test is done in order to test the function of Forgot Password form for any kind of user of the “bike4share” webapp.
\subsection{Test Items}
\begin{itemize}
    \item Forgot Password form on the page /forgotpassword
\end{itemize}
\subsection{Features to be Tested}
\begin{itemize}
    \item The request for the Forgot Password is successfully complete only if the user receive an email with a specific code that will be used in the Set New Password form.
\end{itemize}
\subsection{Features Not to Be Tested}
\begin{itemize}
    \item A minimum number of character for the e-mail address is not required.
    \item The domain of the e-mail 
\end{itemize}
\subsection{Approach}
\begin{itemize}
    \item The test is done manually. And the different combination described into the schedule are done. 
\end{itemize}
\subsection{Item Pass/Fail Criteria}
\begin{itemize}
    \item The test is passed if the user receive an e-mail with the code and that code is registered into the database of password\textunderscore recovery.
   \end{itemize}
\subsection{Schedule}
\begin{itemize}
    \item The form is filled with an not existing e-mail
    \item The form is not filled 
\end{itemize}
\subsection{Final State}
\begin{itemize}
    \item The test is done manually and all the features tested works.
\end{itemize}

\section{Set New Password}
\subsection{Introduction}
The test is done in order to test the function of the Set New Password form for any kind of user of the “bike4share” webapp.
\subsection{Test Items}
\begin{itemize}
    \item Set New Password form on the page/set\textunderscore new\textunderscore password
\end{itemize}
\subsection{Features to be Tested}
\begin{itemize}
    \item The new password is set only if the username and the code sent by email to the user are written and the code is corrected.
    \item The Set New Password procedure doesn’t works for an incorrect username or code  
\end{itemize}
\subsection{Features Not to Be Tested}
\begin{itemize}
    \item A minimum number of character for username and the new password is not required.
    \item the identity between the new and the old password is not checked.
\end{itemize}
\subsection{Approach}
\begin{itemize}
    \item The test is done manually. And the different combination described into the schedule are done. 
\end{itemize}
\subsection{Item Pass/Fail Criteria}
\begin{itemize}
    \item The test is passed if the user with the code received by email and the correct username is able to set a new password and it is registered into the database of bike\textunderscore users, the new password is registered in the user\textunderscore password column instead of the old one and the code received by email is deleted from the database password\textunderscore recovery.
   \end{itemize}
\subsection{Schedule}
\begin{itemize}
    \item The form is filled without putting the received code. Only username “user1” and new password “user1pa” are inserted.
    \item The form is filled without putting the new password. Only received code and username “user1”are inserted.
    \item The form is filled without putting the username. Only received code and password “user1pa" are inserted.
    \item The form is filled with received code, username “user1” and password “user1pa”.
    \item The form is filled with an not existing username “test”.
    \item The form is filled with an invented code.
\end{itemize}
\subsection{Final State}
\begin{itemize}
    \item The test is done manually and all the features tested works.
\end{itemize}

\end{document}