\documentclass{article}
\usepackage[utf8]{inputenc}
\usepackage{graphicx}
\usepackage{float}
\usepackage{mathtools}

\title{Test plan document: bike4share}
\author{Maffioli Sara, Papale Lorenzo, \\ Stucchi Lorenzo \& Vaghi Federica}


\begin{document}
\maketitle
\tableofcontents

\newpage

\section{User Registration}
\subsection{Introduction}
The test is done to test the function of the registration form for a normal user of the “bike4share” webapp.
\subsection{Test Items}
\begin{itemize}
    \item Registration form on the page /register
\end{itemize}
\subsection{Features to be Tested}
\begin{itemize}
    \item The registration is done only if the username and the password are written
    \item The registration doesn’t works for an already existed username
\end{itemize}
\subsection{Features Not to Be Tested}
\begin{itemize}
    \item A minimum number of charter for username and password is not required. 
\end{itemize}
\subsection{Approach}
\begin{itemize}
    \item The test is done manually. And the different combination described into the schedule are done. 
\end{itemize}
\subsection{Item Pass/Fail Criteria}
\begin{itemize}
    \item The test is passed if only the user with a new username and a password is registered into the database of bike\textunderscore users and the user is registrated with user\textunderscore type equal to u (normal user). 
\end{itemize}
\subsection{Schedule}
\begin{itemize}
    \item The form is filled without putting a password. Only username “test 1”
    \item The form is filled without putting a username. Only password “testpa”
    \item The form is filled with username “test 1” and password “testpa”
    \item The form is filled with an existing username “ test1” .
\end{itemize}
\subsection{Final State}
\begin{itemize}
    \item The test is done manually and all the features tested works.
\end{itemize}


\section{Technician Registration}
\subsection{Introduction}
The test is done to test the function of the registration form for a techician user of the “bike4share” webapp.
\subsection{Test Items}
\begin{itemize}
    \item Registration form on the page /tec\textunderscore reg
\end{itemize}
\subsection{Features to be Tested}
\begin{itemize}
    \item The registration is done only if the secret code, the username and the password are written and the secret code is not yet used.
    \item The registration doesn’t works for an already used secret code or existed username
\end{itemize}
\subsection{Features Not to Be Tested}
\begin{itemize}
    \item A minimum number of character for username and password is not required. 
\end{itemize}
\subsection{Approach}
\begin{itemize}
    \item The test is done manually. And the different combination described into the schedule are done. 
\end{itemize}
\subsection{Item Pass/Fail Criteria}
\begin{itemize}
    \item The test is passed if only the user with an never used secret code, a new username and a password is registered into the database of bike\textunderscore users, the user is registered with user\textunderscore type equal to t (technician user) and the secret code is deleted from the database secret\textunderscore key.
   \end{itemize}
\subsection{Schedule}
\begin{itemize}
    \item The form is filled without putting a secret code. Only username “tec1” and password “tecpa”.
    \item The form is filled without putting a password. Only secret code and username “tec1”
    \item The form is filled without putting a username. Only secret code and password “tecpa”
    \item The form is filled with secret code, username “tec1” and password “tecpa”.
    \item The form is filled with an existing username “ test1”.
    \item The form is filled with an invented secret code.
    \item The form is filled with an already used secret code.
\end{itemize}
\subsection{Final State}
\begin{itemize}
    \item The test is done manually and all the features tested works.
\end{itemize}

\end{document}